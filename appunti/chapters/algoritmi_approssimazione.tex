\section{Algoritmi di approssimazione}

Recap problemi di ottimizzazione, pag 42.
TODO

Definizione algoritmo di rho-approssimazione, pag 43.
TODO

Definizione schema di approssimazione, pag 44.
TODO

Definizione schema di approssimazione polinomiale, pag 44.
TODO

Definizione schema di approssimazione pienamente polinomiale, pag 44.
TODO

\section{Vertex cover}
% pag 45
Approssimazione del VC, pagg 45-48.
TODO

\section{Pezzi utili di \LaTeX{}}
\begin{algorithm}[H]
\caption{Divide and Conquer}\label{alg:dnc}
\begin{algorithmic}[1]
    \Procedure{D\&C}{$i$}
        \If{$|i| \leq n_0$}
        \Comment{BASE}
            \State *risolvo direttamente*
        \EndIf
        \State $\langle i_1, i_2, \dots, i_k \rangle \gets A_D(i)$ 
        \Comment{DIVIDE}
        \For{$j \gets 1 $ to $ k $ }
        \Comment{RECURSE}
            \State $s_j \gets $ \Call{D\&C}{$i_j$}
        \EndFor
        \State $s \gets A_C(\langle s_1, s_2, \dots, s_k \rangle)$
        \Comment{CONQUER}
        \State return $s$
    \EndProcedure
\end{algorithmic}
\end{algorithm}
\noindent
Testo non identato!

\begin{definition}[Algoritmo]\label{def:algex}
    Un algoritmo è una procedura computazionale finita (terminante) e deterministica, specificata come una sequenza di passi elementari (istruzioni) estratte da un insieme standard associato a un modello computazionale (astrazione di un computer) che trasforma in maniera univoca un ingresso in un uscita.
\end{definition}

Guarda che so fare
\begin{equation*}
    \setzo{m}
    \quad
    \setzo{}
\end{equation*}

Un problema
\begin{align*}
    SS: & \\
    \texttt{istanza:} \quad & \langle S,t \rangle \\
    \text{dove} \quad & S \subseteq \mathbb{N} - \left\{ 0 \right\} \text{ finito} \\
    & t \subseteq \mathbb{N} - \left\{ 0 \right\} \\
    \texttt{domanda:} \quad & \exists \, S' \subseteq S : \sum_{s \in S'}^{} s = t \, ?
\end{align*}

Una lista
\begin{itemize}[noitemsep,parsep=0pt,partopsep=0pt,topsep=0pt]
    \item[--] $L_A = L$ (il linguaggio deciso da $A$ è $L$)
    \item[--] $T_A(|x|) = O(|x|^k)$ per qualche costante $k \geq 0$
\end{itemize}
